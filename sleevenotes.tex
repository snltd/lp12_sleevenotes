\documentclass[a3paper]{article}
\usepackage{multicol}
\usepackage[paperheight=12in,paperwidth=12in,margin=1in,heightrounded]{geometry}
\usepackage{MnSymbol}
\def\inch#1{#1''}
\def\ft#1{#1'\thinspace}

\iffalse
\newlength{\normalparindent}
\setlength{\normalparindent}{\parindent}
\raggedright
\setlength{\parindent}{\normalparindent}
\fi

\setlength\columnseprule{0.5pt}
\begin{document}
\begin{multicols}{5}

\vspace*{20mm}

\begin{center}
\huge ``LP-12''
\bigskip

\Large Robert Fisher
\end{center}

\bigskip

Work on this series began in early 2009, when vinyl LPs were an anachronistic
  minority interest. Now, in 2019, records are a popular fashion accessory and
  the object of much fetishization. What started out as a pure celebration has
  been soured by contrarianism, as what once seemed worth saying may now be
  just clich\'e.

These paintings, each the size of a 12" LP sleeve, can be seen as tracks in a
  visual album.  Each is a self-contained whole, but is sympathetic to its
  companions. As a unifying concept, every painting makes reference to a
  different artistic movement, and each has an epigram scratched into its
  surface, in the way messages are often scratched into the run-out grooves of
  records.

\begin{center}
*
\end{center}

Our record collections are reflections of our selves.
  \textbf{``\textit{Untitled}''} (\textsc{slaclp01}) is a self-portrait,
  appropriating the curved-mirror motif of \textit{The Arnolfini Portrait}.
  The ``runout groove quote'' says \textit{I hear my needle hit the groove},
  and the album begins. Record player, model's own.

A concentric-groove record has two grooves on the same side: you get different
  songs depending where you drop the needle.  \textbf{``\textit{Heavy Meta}''}
  (\textsc{slaclp02}) is made up of two concentric square spirals. The outer
  spiral contains names of painters and graphic artists whose work has been
  used on LP sleeves in the artist's collection. The inner is a list of
  photographers. Letter colours are random, but for the clicks-and-pops of the
  the white letters which, when read from the top spell out a hidden credit.
  The background is an inverted, pixellated image of a well-known pop group.
  The style is modern, referencing geometric abstraction, grid, and text
  painting, As the run-out says, \textit{classic art has had its day}.

The title and run-out text (\textit{one of the best songs I've ever heard}) of
  \textbf{``\textit{Thriller}''} (\textsc{slaclp03}) contradicts the
  painting's style and literal message.  One  man's ``thriller'' is another
  man's ``filler''.  Records are an equal-opportunity medium, making you
  listen to every song every time. A person gets out of art what they put in,
  and eventually you might come to love something you first thought was
  difficult, or dull. This is ultimately a more rewarding experience than
  hitting skip until your favourite comes on. Like a record, the image was
  made by printing an inverse.

\textbf{``\textit{More than Seven Variables}''} (\textsc{slaclp04}) is a
  collection of constants, formulae and references relating to the
  mathematics, science, and taxonomy of records. The placing of everything is
  important. The $ \diameter $ symbol is central, and exactly the size it
  says: the size of the hole in a record.  Relative to that point, $ \infty $
  is where the never-ending run-out groove should be; \inch{10} is on a
  five-inch radius and so-on.  \textsc{Loud} (referencing Robert Indiana's
  \textit{Love} sculpture) and \textsc{Quiet} are where loud and quiet tracks
  should be on a correctly sequenced LP, as dynamic range decreases on the
  inner bands.  `1/2 3/4` could refer to the sides of a double LP, or given
  its location, a count-in.  The run-out text comes from King Missile's
  \textit{Sensitive Artist}: \textit{working on my work, which no one
  understands}.

\textbf{``\textit{I'm Rebuying my Twenties (180 grams at a Time)}''}
  (\textsc{slaclp05}) was built like a collection: randomly growing with no
  final aim; every square sized, designed and painted in isolation, with no
  plan towards a harmonic whole.  The run out says \textit{oh, you have loads
  of songs}, and the visual cue is from the decorative arts.  It's a rehash of
  an old painting, but replaces the original's colourful chaos with a more
  arch, "curated" feel.  Every square is supposed to be unique, but good
  collections always have the odd duplicate.

Ten years ago you could barely give records away.  Today even hopeless tat
  commands ridiculous prices. Everyone sees the price of unsold listings on
  eBay and Discogs, then thinks theirs is special, and is worth even more.
  They will also likely overstate the condition of their item.
  \textbf{``\textit{VG+}''} (\textsc{slaclp06}), a response to buying a
  ``VG+'' record fit only for the bin,  is a collage made from an utterly
  destroyed record and its sleeve, with some gold leaf (which is also worth
  far less than people think) thrown in to push the value even higher.
  Visually, it suggests catastrophe, and a shelf of records.  Run-out:
  \textit{Let's make lots of money}.

\textbf{``\textit{No Dirty Needles}''} \textsc{slaclp07} is an action painting
  made on a Loricraft PRC-4 professional record-cleaning machine. Cleaning
  records is my therapy. Run-out: \textit{I do it clean}.

Many of the components of \textsc{slaclp04} are related. For instance, `A' and
  `b', or the four Kanji which represent the colours of cartridge wires.
  Connecting related points produces a pattern of triangles, which became an
  abstract landscape.  Titled \textbf{``\textit{No Meaning but Your Own
  Meaning}''} \textsc{slaclp08}, it's a remix. And it's geometry.
  \textit{Come disconnect the dots with me}.

\textbf{``\textit{Untitled}''} \textsc{slaclp09} When most people talk of
  sleeve-art, they soon mention gatefolds, a staple of '70s and '80s rock.
  This figure painting tries to fit an '80s gatefold (centrefold?) into a
  single panel, and fails. Does not currently have a run-out text.

\textbf{``\textit{Untitled}''} \textsc{slaclp10} uses Pop-Art's tropes of
  surface and commerciality to reference the joy of second-hand-record
  shopping.  When so many people hang record sleeves on their walls in lieu of
  art\footnote{the original work used on a record sleeve may be a piece of
  art, but the sleeve itself is packaging},  here is art saying, if the
  conspicuous show of taste is all that matters, why not just put the proof of
  purchase up there? \textit{Never for money, always for love}.

A second painting of a record-player bookends the collection, but
  \textbf{``\textit{After the End}''} (\textsc{slaclp11}) is very much a
  closing piece. The needle sits in the run-out groove; a life-support machine
  keeping the clicking heartbeat going until someone pulls the plug. What
  comes next? This picture already existed, as a painting and a photograph. It
  is a pointless reissue of something no one much cared about before.

\end{multicols}
\end{document}
